\Section{Conclusion}
\label{sec:conclusion}

In this thesis, di-Higgs processes in the $HH \rightarrow b \bar{b} \gamma \gamma$ decay channel were studied, which are directly related to the BEH mechanism and beyond the Standard Model (BSM)
theories through various couplings, such as the trilinear self-coupling of the Higgs boson, $\lambda_{HHH}$. The main production modes for the creation of two Higgs bosons are vector boson fusion (VBF)
and gluon-gluon fusion (ggF). \\

The $b \bar{b} \gamma \gamma$ decay channel is one of the most sensitive channels because the decay of one Higgs boson into two bottom quarks has a large branching ratio, and the two photons produced by the decay of the other Higgs boson
can be detected with high mass resolution. The dominant background processes in this decay channel can be divided into two classes: The first class is the $t \bar{t} H$ background, where two top quarks radiate a Higgs boson that
decays into two photons. If each top quark decays into a vector boson and a b-quark, the resulting final state consists of two photons and two b-jets with kinematics similar to the signal.
The second background class is the non-resonant background, which includes $\gamma$+jets and $\gamma\gamma$+jets events. \\

To distinguish the two di-Higgs production modes from background processes in the $b \bar{b} \gamma \gamma$ decay channel, a multiclass classification using a multi-layer perceptron (MLP) was performed.
The neural network's performance depends on the hyperparameters, which were optimized using both Bayesian optimization and random search algorithms to minimize the loss function.
A comparison of the two methods revealed that Bayesian optimization achieved only a marginally better loss value, indicating that random search is sufficient for this study. \\

The performance of the MLP was evaluated using ROC curves and confusion matrices, showing that the non-resonant background could be most effectively distinguished from the other classes, consistent with
the fact that its kinematics differ the most from the signal. The $t \bar{t} H$ background was also well identified. However, two major sources of
misclassification were identified: First, the model frequently misclassified ggF signal events as $t \bar{t} H$ background, while the misidentification between VBF and $t \bar{t} H$ was less pronounced. This can be explained by the greater kinematic similarity between ggF and $t \bar{t} H$ compared to VBF and $t \bar{t} H$. 
In many VBF production modes, the Higgs bosons tend to be produced along the beam axis, while in the ggF and $t \bar{t} H$ processes, the direction of the Higgs bosons relative to the beam axis
is more arbitrary. Second, the two signal classes, ggF and VBF, were often confused. Increasing the weights of the signal classes during training did not resolve this issue, as seen in the AUC values. \\

To better understand these misclassifications, distributions of ggF events identified as VBF and VBF events identified as ggF were compared to inclusive ggF and VBF samples. It was found that variables
such as the Collins-Soper angle, the $\eta$ distributions of the VBF-tagged jets and the transverse momenta scaled by the invariant masses of the decay products play a major role in the misclassification. \\

Future work could involve calculating the significance, which measures the number of correctly identified signal events relative to background events. Additionally, it would be worthwhile to explore how the
performance of the MLP changes if certain low-level features are removed, as many high-level features are already provided and low-level features might be redundant. Finally, a Graph Attention Network could be
a suitable alternative for classification, as it can dynamically adjust the number of objects per event and learn the relationships between them. A Graph Attention Network was also constructed in this thesis,
but due to time constraints, it could not be studied in detail. The results and further details can be found in the appendix. \\

In conclusion, this thesis demonstrates how an MLP performs in a signal-background classification for the $HH \rightarrow b \bar{b} \gamma \gamma$ decay channel and provides insight into the challenges in
distinguishing between signal and background processes, facilitating future studies for the $HH \rightarrow b \bar{b} \gamma \gamma$ analysis at CMS for Run-3. \\