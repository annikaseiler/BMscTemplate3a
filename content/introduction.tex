\Section{Introduction}
\label{sec:introduction}

Since the discovery of the Higgs boson in 2012, there has been a pronounced interest in understanding the
phenomenology of the new particle and especially the Brout-Englert-Higgs mechanism. One of the parameters
that determines the shape of Higg's potential is the triliniear self-coupling of the Higgs boson $\lambda_{HHH}$,
which is directly accessible via di-Higgs production~\cite{CMS:2021qbp, HandbookLHC}. Of particular interest is the fact that beyond Standard Model (BSM) theories
predict very different di-Higgs behaviour, some of which differ markedly from predictions of the Standard Model (SM).
Contributions from BSM effects may significantly increase di-Higgs production and influence the di-Higgs phenomenology, so providing BSM benchmarks
through studying di-Higgs physics can vastly increase our understanding of the Brout-Englert-Higgs mechanism~\cite{HandbookLHC}. \\

The $HH \rightarrow b \bar{b} \gamma \gamma$ is one of the most sensitive decay channels of di-Higgs, since the decay of one Higgs boson
to two bottom quarks has a large branching fraction and the two photons produced by the second Higgs boson can be detected with a good mass resolution.
Furthermore in this decay channel there occur comparatively few background processes~\cite{CMS:2021qbp}, which makes it easier to distinguish the signal from the backgrounds.
Nevertheless the signal-background classification
is a challenging task, which is performed by constructing physics motivated variables and applying selection cuts on the various objects.
In this thesis the signal-background classification using a multilayer perceptron (MLP) is presented. \\

The thesis is structured as follows. In Section \ref{sec:theory}, the role of the Higgs boson in the SM, HH production
and the CMS detector will be introduced. In Section \ref{sec:prep}, the event selection and variables used to distinguish signal from background will be presented.
The architecture of the MLP is introduced in Section \ref{sec:deepl} and in section \ref{sec:results} the results of the signal-background classification using the MLP are presented.