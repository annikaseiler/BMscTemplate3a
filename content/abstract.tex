\section*{Abstract}

This work presents a signal-background classification for the $HH \rightarrow b \bar{b} \gamma \gamma$ search using a multilayer perceptron (MLP).
The two production modes of the Higgs boson pair, gluon-gluon fusion (ggF) and vector boson fusion (VBF), are distinguished from both the associated
top-quark production process $t \bar{t} H$ and the non-resonant background processes, which include $\gamma$+jets and $\gamma \gamma$+jets final states.
The hyperparameters of the MLP are optimized using both Bayesian optimization and random search. The minimal validation loss achieved by both algorithms
shows no significant difference, although Bayesian optimization is a considerably more complex algorithm. \\

The MLP achieves its best performance in identifying the non-resonant background, with an area under the curve (AUC) of $0.9949$ for the one-vs-all
receiver operating characteristic (ROC) curve, where each class is compared against all others. This is followed by its performance in identifying the $t \bar{t} H$ background,
where an AUC of $0.9768$ is obtained.
For the ggF and VBF signal classes, AUC values of $0.9391$ and $0.9495$, respectively, are achieved. The primary challenge lies in distinguishing between the two
signal classes (ggF and VBF).
Adjusting the training weights for the signal classes does not lead to any notable improvement. \\

When analyzing the misclassifications — where gluon-gluon fusion events are identified as vector boson fusion and vice versa — it becomes evident
that the distribution of the Collins-Soper angle, the $\eta$ distribution of the VBF jets, the angular separations between photon and b-jet candidates, and various
kinematic properties of the photons and b-jets are significant factors contributing to the misidentification.