\Section{HH production in the $b \bar{b} \gamma \gamma$ decay channel}
\label{sec:prep}

In this thesis HH production is studied in the $b \bar{b} \gamma \gamma$ decay channel, where one Higgs boson decays into two photons
and the other one into a b-jet pair. In the following chapter the background processes and event selection criteria regarding this decay channel are described.

\subsection{Background processes}
\label{sec:bkgproccess}

The most dominant background process is the production of a single Higgs boson by a Top-Quark Antitop-Quark pair ($t\bar{t}H$), the feynman graph can be seen in Fig. \ref{fig:3}.
Each Top-Quark decays with a branching ratio of $99.7 \%$ into a Bottom-Quark and a W-Boson, so if the Higgs boson decays into a photon pair
it is very likely to observe a $b \bar{b} \gamma \gamma$ final state. In addition the two b-jets show very similar kinematics compared to the 
b-jets of the signal events, which is why the $t\bar{t}H$-background leaves a very similar track in the detector compared to the signal.
The processes can best be distinguished by the decay of the W-Bosons in $t\bar{t}H$: With a probability of $33 \%$ each W-Boson decays into
a lepton and a neutrino. Not only are there no leptons produced in the signal events, but since neutrinos are often not detected the processes
show different missing transverese-energy (MET) distributions. \\

%\Figure{H}{0.8}{fig:3}{Feynman graphs ttH}{ - The production modes of ttH.}{content/Plots/feynmangrah_ttH.png}

Finally $\gamma  + jets$ and $\gamma \gamma  + jets$ events represent the last imortant background category, exemplary feynman graphs are shown in Fig. ???.
In general the photons and jets from these events are located in a lower energy range than the signal. If the photons and jets are decay products of a Higgs boson,
as is the case in the signal processes and in the $t \bar{t} H$-background, the invariant mass of the photon pair or jet pair must be just as large as the Higgs mass. In the $\gamma  + jets$ and $\gamma \gamma  + jets$
background process, by contrast, the decay products can be radiated from particles with far less energy, which means that they themselves can also have lower energies. 
In addition the spatial distriutions of the photons and jets are wider, since they don't necessarily are decay products
of the same particle. Thats why in general the $\gamma  + jets$ and $\gamma \gamma  + jets$ can be better differentiated from the signal than
the $t\bar{t}H$-background. \\

\subsection{Event selection}
\label{sec:HiggsDNA}

In this thesis Monte Carlo-simulated data are used. The number of events per class can be seen in Table \ref{tab:1e}.

\Table{H}{tab:1e}{Number of events per class}{}{c c}{
    \hline
    Class & Number of events \\
    \hline
    Non-resonant background & 2 133 185  \\
    $t \bar{t} H$ background & 178 694  \\
    ggF to HH signal & 104 354 \\
    VBF to HH signal & 558 685 \\
    \hline
}

For each event a diphoton pair, a dijet pair and two VBF jets are reconstructed. The diphoton pair is built by the two photons with the highest $p_T$, the dijet pair is constructed by using the two jets
with the highest b-tagging scores and the VBF jet pair is built with the two jets with the highest dijet invariant mass.
For further identification of the diphoton or dijet pair, selection cuts based on the HiggsDNA are applied to the variables. The selection
cuts for the diphoton-pair are presented in Table \ref{tab:2e}, the ones for the dijet-pair in \ref{tab:3e} and the ones for the VBF jets in \ref{tab:4e}. \\

\Table{H}{tab:2e}{Diphoton selection cuts}{}{c c}{
    \hline
    Variable & Cut \\
    \hline
    leading $p_T$ & > $35$ GeV  \\
    subleading $p_T$ & > $25$ GeV  \\
    $\frac{p_T^{\gamma_1}}{M_{\gamma \gamma}}$ & > $\frac{1}{3}$ \\
    $\frac{p_T^{\gamma_2}}{M_{\gamma \gamma}}$ & > $\frac{1}{4}$ \\
    Barrel  $| \eta |$ & < $1.4442$ \\
    Endcap $| \eta |$ & > $1.566$ \\
    H/E & < $0.08$ \\
    MVA ID & > $-0.9$ \\
    R9 & > $0.8$ \\
    ChI & < $20$ \\
    ChI/$E_T$ & < $0.3$ \\
    \hline
}

\Table{H}{tab:3e}{Dijet selection cuts}{}{c c}{
    \hline
    Variable & Cut \\
    \hline
    $p_T$ & > $20$ GeV  \\
    $M_{bb}$ & > $70$ GeV \\
    $M_{bb}$ & < $190$ GeV \\
    $| \eta |$ & < $2.5$ \\
    $\Delta R_{b, \gamma \gamma}$ & < $0.4$ \\
   $\Delta R_{b, \gamma}$ & < $0.4$ \\
    \hline
}

\Table{H}{tab:4e}{VBF jets selection cuts}{}{c c}{
    \hline
    Variable & Cut \\
    \hline
    leading $p_T$ & > $40$ GeV  \\
    subleading $p_T$ & > $30$ GeV  \\
    $| \eta |$ & < $4.7$ \\
    $\Delta R_{j, \gamma}$ & < $0.4$ \\
    $\Delta R_{j, b}$ & < $0.4$ \\
    \hline
}

There are multiple reasons to apply selection cuts on the variables:
First the detector resolution is important to consider by implementing cuts on $| \eta |$, since using data from an area where the resolution is bad could falsify the results.
Second for some kinematic variables it is already known in what range they should be. For example should the invariant masses of the diphoton- and dijet-pair match the Higgs mass and because of
an uncertainty due to the detector's resolution and statistic fluctuation the corresponding variables should be in an interval around the Higgs mass.
Third it makes sense to apply cuts on identification variables like the photon MVA ID and b-tagging scores so that only variables with a sufficiently high probability of actually being the corresponding object are used. 
Finally a certain signature left in the detector is expected for the objects from signal events. For example for a photon there should be much more energy deposited in the ECAL than in the HCAL and 
the angular distance between a b-jet and a photon being decay products of two different Higgs bosons should not be too close.
Applying all these cuts ensures that the correct objects are actually identified as a diphoton or dijet pair so that the performance of the neural networks is improved.

\subsection{Variables to distinguish signal from background}
\label{sec:trainvar}

The choice of the variables used for signal-background classification depends on the neural network that is used. For the MLP the variables can mostly be grouped into three
categories: Kinematic variables, object resolution variables and object identification variables~\cite{CMS:2021qbp}. Each event can have, depending on the underlying process, two photons, two b-jets,
two VBF-jets, up to six jets in total and up to four leptons. For all these objects several variables which are suitable to distinguish signal from background are used.
In Table \ref{tab:6e} the variables regarding the photons, b-jets and VBF-jets are listed~\cite{Run2analysisnote}.

\Table{H}{tab:6e}{Photon, b-jet and VBF-jet variables}{}{c c c c}{
    \hline
    Description & Photons & b-jets & VBF-jets \\
    \hline
    lead and sublead & $p_T^{\gamma}/m^{\gamma \gamma}$ & $p_T^{b}/m^{bb}$ & $p_T^{j}/m^{jj}$ \\
    & $p_T^{\gamma \gamma}$/$m^{\gamma \gamma bb}$ & $p_T^{bb}$/$m^{\gamma \gamma bb}$ & \\
    lead and sublead & $p_T^{\gamma}$ & $p_T^{b}$ & $p_T^{j}$\\
    lead and sublead & $\eta^{\gamma}$ & $\eta^{b}$ & $\eta^{j}$\\
    lead and sublead & $\phi^{\gamma}$ & $\phi^{b}$ & $\phi^{j}$\\
    lead and sublead & MVA ID & b-tag & VBF-tag \\
    \hline
}

In addition for the jets one to six and the leptons one to four the variables $p_T$, $\eta$, $\phi$ are used. Also the generation for each lepton is stated and the number of leptons and jets is
also given as an input. Moreover the variables $\sigma_M/M$, which is the mass resolution for the diphoton pair and $\rho$, which is the median energy density of the jets, are stated. \\

Furthermore angular variables between different objects are also taken into account. The angular distances $\Delta R$ between each photon and jet of the diphoton and dijet pair 
and between the two VBF jets and each of the two photons and b-jets are used, together with the minimal angular distance between a photon and a b-jet ($\Delta R_{min}^{\gamma b}$), between a VBF-jet and a photon
($\Delta R_{min}^{j \gamma}$) and between a VBF-jet and a b-jet ($\Delta R_{min}^{j b}$).
Also the helicity angles $cos(\theta_{\gamma \gamma})$ between the two photons, $cos(\theta_{bb})$ between the two b-jets and the Collins-Soper angle $cos(\theta_{CS}^*)$ are stated.
The Collins-Soper angle is the angle between the $H \rightarrow \gamma \gamma$ candidate~\cite{Run2analysisnote} and the Collins-Soper reference frame, where the z-axis goes along the beam axis~\cite{CSangle}.
Moreover the MET-variables $MET \phi$, $MET p_T$, $\Delta \phi (MET, b_1)$ and $\Delta \phi (MET, b_2)$ are used. The $M_X$ variable is stated for the four body mass,

\eqn{
    M_X = M(bb \gamma \gamma) - M(bb) - M(\gamma \gamma) + 2M_H
}

where $M_H = 125$ GeV is the Higgs mass~\cite{Run2analysisnote}. To reject those events from the $t \bar{t}H$-background, where a W-Boson decays into a quark-antiquark pair, the $\chi_{t0}$-variable for events between four and six jets,

\eqn{
    \chi_{t0}^2 = \bigl( \frac{m_w-m_{jj}}{0.1 \cdot m_w} \bigr)^2 +  \bigl( \frac{m_t-m_{bjj}}{0.1 \cdot m_t}\bigr)^2
}

and the $\chi_{t1}$-variable for events with six jets or more is used,

\eqn{
    \chi_{t1}^2 = \bigl(\frac{m_w - m_{jj, 1}}{0.1 \cdot m_w} \bigr)^2 + \bigl(\frac{m_t-m_{bjj, 1}}{0.1 \cdot m_t} \bigr)^2 + \bigl(\frac{m_w - m_{jj, 2}}{0.1 \cdot m_w} \bigr)^2 + \bigl(\frac{m_t-m_{bjj, 2}}{0.1 \cdot m_t} \bigr)^2
}

where $m_w=80.377$ GeV is the mass of a W-boson and $m_t=172.76$ GeV is the topquark's mass~\cite{Run2analysisnote}.
The centrality variable that relates the direction of the VBF-jets to that of the Higgs candidates is defined as follows,

\eqn{
    C_H = exp \biggl[ - \frac{4}{(\eta_1^{VBF} - \eta_2^{VBF})^2} \bigl( \eta^H - \frac{\eta_1^{VBF} + \eta_2^{VBF}}{2} \bigr)^2 \biggr]
}

for both of the Higgs candidates this is used as an input~\cite{CMS:2021qbp}. \\


% "chi_t0",
% "chi_t1",

% "VBF_first_jet_btagPNetQvG",
% "VBF_second_jet_btagPNetQvG",

% "VBF_jet_eta_prod",
% "VBF_jet_eta_diff",
% "VBF_jet_eta_sum",

% "VBF_Cgg",
% "VBF_Cbb"