\Section{HH production in the $b \bar{b} \gamma \gamma$ decay channel}
\label{sec:prep}

In this thesis HH production is studied in the $b \bar{b} \gamma \gamma$ decay channel, where one Higgs boson decays into two photons
and the other one into a b-jet pair. In the following chapter the background processes, event selection criteria regarding this decay channel and the variables 
used as an input for the deep neural network are described.

\subsection{Background processes}
\label{sec:bkgproccess}

One of the dominant background processes is the production of a single Higgs boson in association with a top antitop quark pair ($t\bar{t}H$), the feynman graph can be seen in Fig. \ref{fig:16}.
Each top quark decays with a branching ratio of $99.7 \%$ into a bottom quark and a W boson~\cite{ParticleDataGroup:2024cfk}, so if the Higgs boson decays into a photon pair
it is very likely to observe a $b \bar{b} \gamma \gamma$ final state. In addition the two b-jets show very similar kinematics compared to the 
b-jets of the signal events, which is why the $t\bar{t}H$-background leaves a very similar signature in the detector compared to the signal.
The processes can best be distinguished by the decay of the W bosons in $t\bar{t}H$: With a probability of $33 \%$ each W boson decays into
a lepton and a neutrino~\cite{ParticleDataGroup:2024cfk}. Not only are there no leptons produced in the signal events, but since neutrinos are often not detected the processes
show different missing transverese-energy (MET) distributions. \\

\Figure{H}{0.3}{fig:16}{Feynman graph for associated top pair production}{ - Exemplary production mode of $t\bar{t}H$ - Image taken from~\cite{CMS:2022Higgs}}{content/Plots/Feynman_ttH.png}
\Figure{H}{0.6}{fig:17}{Exemplary Feynman graphs for non-resonant background}{}{content/Plots/Feynman_non_resonant.png}

Finally $\gamma  + jets$ and $\gamma \gamma  + jets$ events represent another imortant background category, exemplary feynman graphs are shown in Fig. \ref{fig:17}.
These events are a background when the occuring photons and jets are misidentified as isolated photons and b-jets. 
As the  $\gamma  + jets$ and $\gamma \gamma  + jets$ events show similar kinematics, they are grouped together as non-resonant background in the following.
% In general the photons and jets from these events are located in a lower energy range than the signal. If the photons and jets are decay products of a Higgs boson,
% as is the case in the signal processes and in the $t \bar{t} H$-background, the invariant mass of the photon pair or jet pair must be just as large as the Higgs mass. In the $\gamma  + jets$ and $\gamma \gamma  + jets$
% background process, by contrast, the decay products can be radiated from particles with far less energy, which means that they themselves can also have lower energies. 
% In addition the spatial distriutions of the photons and jets are wider, since they don't necessarily are decay products
% of the same particle. Thats why in general the $\gamma  + jets$ and $\gamma \gamma  + jets$ can be better differentiated from the signal than
% the $t\bar{t}H$-background. \\

\subsection{Event selection}
\label{sec:HiggsDNA}

In this thesis Monte Carlo-simulated data are used. 

In order to remove events, for which it is clear from prior knowledge about the distribution of certain
variables that they cannot be signal events, from the data set, that are used as input for the neural network, selection cuts based on the HIggsDNA are applied in the following.

For each event a diphoton pair and a dijet pair is reconstructed. The diphoton pair is built by the two photons with the highest $p_T$ and the dijet pair is constructed by using the two jets
with the highest b-tagging scores.
Furthermore after the diphoton and dijet pair is built, a VBF jet pair is selected for each event, where the VBF jets are the colliding quarks from the VBF production process. The VBF jet pair is built with the two jets with the highest
dijet invariant mass. 
The selection cuts for variables regarding the diphoton-pair are presented in Table \ref{tab:2e}, the ones for variables regarding the dijet-pair in \ref{tab:3e} and the ones for variables regarding the VBF jets in \ref{tab:4e}. \\

\Table{H}{tab:2e}{Photon selection}{}{c c}{
    \hline
    Variable & Cut \\
    \hline
    leading $p_T$ & > $35$ GeV  \\
    subleading $p_T$ & > $25$ GeV  \\
    $\frac{p_T^{\gamma_1}}{M_{\gamma \gamma}}$ & > $\frac{1}{3}$ \\
    $\frac{p_T^{\gamma_2}}{M_{\gamma \gamma}}$ & > $\frac{1}{4}$ \\
    $M_{\gamma \gamma}$ & > $100$ GeV \\
    $M_{\gamma \gamma}$ & < $180$ GeV \\
    Barrel  $| \eta |$ & < $1.4442$ \\
    Endcap $| \eta |$ & > $1.566$ \\
    H/E & < $0.08$ \\
    MVA ID & > $-0.9$ \\
    \hline
}

% R9 & > $0.8$ ChI & < $20$ ChI/$E_T$ & < $0.3$

\Table{H}{tab:3e}{Jet selection}{}{c c}{
    \hline
    Variable & Cut \\
    \hline
    $p_T$ & > $20$ GeV  \\
    $M_{bb}$ & > $70$ GeV \\
    $M_{bb}$ & < $190$ GeV \\
    $| \eta |$ & < $2.5$ \\
    $\Delta R_{b, \gamma}$ & > $0.4$ \\
    \hline
}

\Table{H}{tab:4e}{VBF jets selection}{}{c c}{
    \hline
    Variable & Cut \\
    \hline
    leading $p_T$ & > $40$ GeV  \\
    subleading $p_T$ & > $30$ GeV  \\
    $| \eta |$ & < $4.7$ \\
    $\Delta R_{j, \gamma}$ & > $0.4$ \\
    $\Delta R_{j, b}$ & >$0.4$ \\
    \hline
}

In addition for the diphoton pair selection criteria on shower shape and isolation variables are applied. The $R_9$ variables is defined as the sum of transverse energy in a 3x3 crystal
array centred around the crystal that is most energetic divided by the energy of the whole supercluster. The charged hadron isolation variable $ChI$ is obtained by summing the transverse momenta
of the selected photon inside an isolation cone of $\Delta R = 0.3$ with respect to the photon's direction~\cite{CMS:2021egamma}. The selection criteria for these variables are, that either $R_9 > 0.8$, $ChI < 20$ GeV or $ChI/E_T < 0.3$
needs to be fulfilled~\cite{Run2analysisnote}, to leave out events where one of the selected photons is not isolated. \\

% There are multiple reasons to apply selection cuts on the variables:
% First the detector resolution is important to consider by implementing cuts on $| \eta |$, since using data from an area where the resolution is bad could falsify the results.
% Second for some kinematic variables it is already known in what range they should be. For example should the invariant masses of the diphoton- and dijet-pair match the Higgs mass and because of
% an uncertainty due to the detector's resolution and statistic fluctuation the corresponding variables should be in an interval around the Higgs mass.
% Third it makes sense to apply cuts on identification variables like the photon MVA ID and b-tagging scores. These variables indicate how strongly an object was identified as a photon or a b-jet.
% The larger the value is, the higher is the probability that the object is actually a photon or a b-jet, so only variables with a sufficiently high probability of actually being the corresponding object are used. 
% Finally a certain signature left in the detector is expected for the objects from signal events. For example for a photon there should be much more energy deposited in the ECAL than in the HCAL and 
% the angular distance between a b-jet and a photon being decay products of two different Higgs bosons should not be too close.

Applying all these cuts improves the correct identification of objects as a diphoton or dijet pair, so that background processes are reduced from the input for the deep neural network.
The number of events per class after applying the selection cuts can be seen in Table \ref{tab:1e}.

\Table{H}{tab:1e}{Number of events per class}{}{c c}{
    \hline
    Class & Number of events \\
    \hline
    Non-resonant background & 2 133 185  \\
    $t \bar{t} H$ background & 178 694  \\
    ggF to HH signal & 104 354 \\
    VBF to HH signal & 558 685 \\
    \hline
}


\subsection{Variables to distinguish signal from background}
\label{sec:trainvar}

% Each event can have, depending on the underlying process, two photons, two b-jets, two VBF-jets, up to six jets in total and up to four leptons.

The input variables for the MLP are grouped into three
categories: Kinematic variables, object resolution variables and object identification variables~\cite{CMS:2021qbp}.
For most of the objects occuring in the events several variables which are suitable to distinguish signal from background are used.
In Table \ref{tab:6e} the variables regarding the photons, b-jets and VBF-jets are listed~\cite{Run2analysisnote}.

\Table{H}{tab:6e}{Photon, b-jet and VBF-jet variables}{}{c c c c}{
    \hline
    Description & Photons & b-jets & VBF-jets \\
    \hline
    lead and sublead & $p_T^{\gamma}/m^{\gamma \gamma}$ & $p_T^{b}/m^{bb}$ & $p_T^{j}/m^{jj}$ \\
    & $p_T^{\gamma \gamma}$/$m^{\gamma \gamma bb}$ & $p_T^{bb}$/$m^{\gamma \gamma bb}$ & \\
    lead and sublead & $p_T^{\gamma}$ & $p_T^{b}$ & $p_T^{j}$\\
    lead and sublead & $\eta^{\gamma}$ & $\eta^{b}$ & $\eta^{j}$\\
    lead and sublead & $\phi^{\gamma}$ & $\phi^{b}$ & $\phi^{j}$\\
    lead and sublead & MVA ID & b-tag & VBF b-tag \\
    \hline
}


In addition for the jets one to six and the leptons one to four the variables $p_T$, $\eta$, $\phi$ are used, where these objects are ordered according to $p_T$. Also the generation for each lepton is stated, which can be electron or muon,
and the number of leptons and jets for each event is
also given as an input. Moreover the variables $\sigma_M/M$, which is the mass resolution for the diphoton pair and for the jets $\rho$, which is the median of the transverse energy density per unit area in the event, are provided as input to the model. \\

Furthermore angular variables between different objects are also taken into account. The angular distances $\Delta R$ between each photon and jet of the diphoton and dijet pair 
and between the two VBF jets and each of the two photons and b-jets are used, together with the minimal angular distance between a photon and a b-jet ($\Delta R_{min}^{\gamma b}$), between VBF-jets and photons
($\Delta R_{min}^{j \gamma}$) and between VBF-jet and a b-jets ($\Delta R_{min}^{j b}$), where the photons and b-jets are part of the diphoton and dijet system. 

Also the helicity angles $cos(\theta_{\gamma \gamma})$ between the two photons, $cos(\theta_{bb})$ between the two b-jets and the Collins-Soper angle $cos(\theta_{CS}^*)$ are stated.
The Collins-Soper angle is the angle between the $H \rightarrow \gamma \gamma$ candidate~\cite{Run2analysisnote} and the Collins-Soper reference frame, where the z-axis goes along the beam axis~\cite{CSangle}.
In Fig. \ref{fig:20} it can be seen that the Higgs boson from the VBF-jet is radiated closely to the beam, whilst in ggF and $t \bar{t} H$ the direction of the $H \rightarrow \gamma \gamma$ candidate is arbitrary.
Moreover the MET-variables, $\phi$ of MET, $p_T$ of MET, $\Delta \phi (MET, b_1)$ and $\Delta \phi (MET, b_2)$ are used. If the $\Delta \phi (MET, b_1)$ variables distribution peaks at zero it means 
that the MET mostly comes from an incorrect energy estimation of the considered object, whilst a wider distribution indicates particles that were not detected, e. g. neutrinos which can occur
in the $t \bar{t} H$ background by the decay of the W bosons. In Fig. \ref{fig:21} it can be seen that the distribution is the widest for the two background classes.


The $M_X$ variable is stated for the four body mass,

\eqn{
    M_X = M(bb \gamma \gamma) - M(bb) - M(\gamma \gamma) + 2M_H
}

where $M_H = 125$ GeV is the Higgs mass~\cite{Run2analysisnote}. To reject those events from the $t \bar{t}H$-background, where a W-Boson decays into a quark-antiquark pair, the $\chi_{t0}$-variable for events between four and six jets,

\eqn{
    \chi_{t0}^2 = \bigl( \frac{m_w-m_{jj}}{0.1 \cdot m_w} \bigr)^2 +  \bigl( \frac{m_t-m_{bjj}}{0.1 \cdot m_t}\bigr)^2
}

is used, where $m_w=80.377$ GeV is the mass of a W-boson and $m_t=172.76$ GeV is the topquark's mass~\cite{Run2analysisnote}.
If one of the top quark decays into a W boson and a b-quark, and the W boson then decays further into a quark-antiquark pair, the according masses will align. So this variable 
will be $0$, if there actually is a quark-antiquark decay of the W boson in the $t \bar{t}H$-background. The distribution of $\chi_{t0}$ is shown in Fig. \ref{fig:22}, where it can
be seen that the distribution for the $t \bar{t}H$-background is in fact the most slim.

For events with six jets or more and the $\chi_{t1}$-variable is used.

\eqn{
    \chi_{t1}^2 = \bigl(\frac{m_w - m_{jj, 1}}{0.1 \cdot m_w} \bigr)^2 + \bigl(\frac{m_t-m_{bjj, 1}}{0.1 \cdot m_t} \bigr)^2 + \bigl(\frac{m_w - m_{jj, 2}}{0.1 \cdot m_w} \bigr)^2 + \bigl(\frac{m_t-m_{bjj, 2}}{0.1 \cdot m_t} \bigr)^2
}

This variable is used to identify both W bosons decay into a quark-antiquark pair, so the terms of $\chi_{t0}$ are applied for the decay products of both W bosons and therefore for both top quarks.
The centrality variable that relates the direction of the VBF-jets to that of the Higgs candidates is defined as follows

\eqn{
    C_H = exp \biggl[ - \frac{4}{(\eta_1^{VBF} - \eta_2^{VBF})^2} \bigl( \eta^H - \frac{\eta_1^{VBF} + \eta_2^{VBF}}{2} \bigr)^2 \biggr]
}

For both of the Higgs candidates this is used as an input~\cite{CMS:2021qbp}, where $C_{bb}$ is the centrality ariable for the two b-jets and $C_{\gamma \gamma}$ is the variable for the diphoton candidate.
This variable will be $1$ when the Higgs boson is radiated into the same $\eta$-region as the VBF jets. Since this is a characteristic
behaviour for VBF events this variable is stated to identify these events. \\

For selected variables the distributions are presented.

\begin{figure}[htbp]
    \centering
    \begin{subfigure}{0.45\textwidth}
        \centering
        \includegraphics[width=\linewidth]{content/Plots/pholead_PtOverM_plot.png}
        \caption{Histogram of the leading $p_T^{\gamma}/m^{\gamma \gamma}$}
        \label{fig:roc1}
    \end{subfigure}
    \hfill
    \begin{subfigure}{0.45\textwidth}
        \centering
        \includegraphics[width=\linewidth]{content/Plots/dijet_PtOverM_ggjj_plot.png}
        \caption{Histogram of $p_T^{bb}$/$m^{\gamma \gamma bb}$}
        \label{fig:roc2}
    \end{subfigure}
    \caption{Histograms of $p_T^{\gamma}/m^{\gamma \gamma}$ and $p_T^{bb}$/$m^{\gamma \gamma bb}$ for all four classes}
    \label{fig:18}
\end{figure}

In Fig. \ref{fig:18}, the histograms of the leading $p_T^{\gamma}/m^{\gamma \gamma}$ and $p_T^{bb}/m^{\gamma \gamma bb}$ are presented. It is evident that the $p_T$ values
scaled by the invariant mass peak in the lowest region for the non-resonant background. This is expected, as the non-resonant background does not involve an highly energetic
Higgs boson, leading to the lowest energy range for the $b \bar{b} \gamma \gamma$ candidate. In VBF events, the Higgs bosons are radiated at a smaller angle to the beam, which
results in lower $p_T$ values compared to the other classes. Consequently, the $p_T$ distributions of ggF and $t \bar{t} H$ are centered at higher $p_T$ values.

\begin{figure}[htbp]
    \centering
    \begin{subfigure}{0.45\textwidth}
        \centering
        \includegraphics[width=\linewidth]{content/Plots/lead_bjet_btagPNetB_plot.png}
        \caption{Histogram of the leading b-jet's b-tagging score}
        \label{fig:roc1}
    \end{subfigure}
    \hfill
    \begin{subfigure}{0.45\textwidth}
        \centering
        \includegraphics[width=\linewidth]{content/Plots/VBF_first_jet_eta_plot.png}
        \caption{Histogram of $\eta$ of the leading VBF jet}
        \label{fig:roc2}
    \end{subfigure}
    \caption{Histograms of the leading b-jet's b-tagging score and of $\eta$ of the leading VBF jet for all four classes}
    \label{fig:19}
\end{figure}

Fig. \ref{fig:19} shows the b-tagging score of the leading b-jet. It can be seen that the distribution of the b-taggng score is centered around a lower value for the non-resonant background than for the other three classes, where 
the b-tagging score shows a similar distribution. This can be explained with the fact that in the non-resoannt background the jets are often misidentified as b-jets, which shows in the b-tagging score.
When studying the $\eta$-distribution of the first VBF jet it can be clearly seen that for the VBF events the absolute value of $\eta$ is larger, since these are the jets originating from the two beam quarks,
whilst for the other classes two random jets with the highest dijet invariant mass are chosen, so they are not mostly radiated close to the beam. \\

The histogram of the minimal angular distance between a photon and a b-jet \ref{fig:20} shows that this variable
has the widest distribution for the two signal classes, since here the photon and the b-jet actually are radiated from two independet Higgs bosons, whilst for the non-resonant background for example
these objects could be radiated in a closer angle to each other.

\begin{figure}[htbp]
    \centering
    \begin{subfigure}{0.45\textwidth}
        \centering
        \includegraphics[width=\linewidth]{content/Plots/DeltaR_jg_min_plot Kopie.png}
        \caption{Histogram of $\Delta R_{min}^{j \gamma}$}
        \label{fig:roc1}
    \end{subfigure}
    \hfill
    \begin{subfigure}{0.45\textwidth}
        \centering
        \includegraphics[width=\linewidth]{content/Plots/absCosThetaStar_CS_plot Kopie.png}
        \caption{Histogram of $cos(\theta_{CS}^*)$}
        \label{fig:roc2}
    \end{subfigure}
    \caption{Histograms of $\Delta R_{min}^{j \gamma}$ and $cos(\theta_{CS}^*)$ for all four classes}
    \label{fig:20}
\end{figure}

\begin{figure}[htbp]
    \centering
    \begin{subfigure}{0.45\textwidth}
        \centering
        \includegraphics[width=\linewidth]{content/Plots/DeltaPhi_j1MET_plot.png}
        \caption{Histogram of $\Delta \phi (MET, b_1)$}
        \label{fig:roc1}
    \end{subfigure}
    \hfill
    \begin{subfigure}{0.45\textwidth}
        \centering
        \includegraphics[width=\linewidth]{content/Plots/n_leptons_plot.png}
        \caption{Histogram of the number of leptons}
        \label{fig:roc2}
    \end{subfigure}
    \caption{Histogram of $\Delta \phi (MET, b_1)$ and of the number of leptons for all four classes}
    \label{fig:21}
\end{figure}

The number of leptons (Fig. \ref{fig:21}) shows that even in the signal classes, the number of leptons is not negligible, which is caused by an inaccurate lepton identification since
there are not any leptons in the signal events. For that reason the leptonic variables are also used as an input for the MLP, because otherwise a large number of events could not be used for classification.

\begin{figure}[!t]
    \centering
    \begin{subfigure}{0.45\textwidth}
        \centering
        \includegraphics[width=\linewidth]{content/Plots/chi_t0_plot.png}
        \caption{Histogram of $\chi_{t0}$}
        \label{fig:roc1}
    \end{subfigure}
    \hfill
    \begin{subfigure}{0.45\textwidth}
        \centering
        \includegraphics[width=\linewidth]{content/Plots/VBF_Cbb_plot.png}
        \caption{Histogram of $C_{bb}$}
        \label{fig:roc2}
    \end{subfigure}
    \caption{Histograms of $\chi_{t0}$ and $C_{bb}$ for all four classes}
    \label{fig:22}
\end{figure}

In Fig. \ref{fig:22} it can be clearly observed that the distribution of $C_{bb}$ peaks at one for the VBF events, whilst this variable is more equally distributed for the other three classes.